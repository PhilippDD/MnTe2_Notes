\documentclass{article}

\usepackage{a4wide}
\usepackage{amsmath}

\title{Better-than-minimal model of VB in MnTe}
\author{Philipp Ritzinger, Karel V\'yborn\'y}
\date{Aug12, 2022}

%%% fte-list
%
%
% eq-01, eq-02

\begin{document}

\maketitle

\section*{Overview}

These notes follow on {\tt mte-notes} and {\tt mt1-notes} and describe
an effective model of the valence band (VB) in MnTe. It considers
%
\begin{equation}
  H=H_{kp}+H_{cryst}+H_{SO}
\label{eq-01}
\end{equation}
%
where $H_{kp}$ is described in~\cite{PFjr} and $H_{SO}$ is (6) in
{\tt mte-notes} --- both $6\times 6$ matrices. Both matrices are
stated below. Parameters of $H_{kp}$ are $a,b,c$ related to the
effective masses simply by comparing e.g. $ck_z^2$ to $\hbar^2 k_z^2/2m_*$:
%
\begin{equation}
  c=\frac{\hbar^2}{2m_*}\qquad \frac{m_*}{m_0}={\textstyle{\frac12}} \frac{\hbar^2}{cm_0}
  \label{eq-02}
\end{equation}  
%
and $e_z\to-\infty$ for the time being (N.B. there's also additional
term, neglected at the moment, i.e. $d=0$). The spin-orbit term is
parametrised by $\Delta_{xy}\not=\Delta_z$ and overlap of (the orbital
parts of) the basis vectors~(3)~of~\cite{PFjr} $t=0$ in the A-point,
i.e. $(k_x,k_y,k_z)=(0,0,0)$, and $t\propto k_z$ as we move along the $A\Gamma$
line.

Next, $H_{cryst}$ (to be found in~\cite{LandT}
as $H_{B-4,g}$) acts effectively as another spin-orbit term albeit of
non-relativistic origin: $J\sigma_x k_zk_x(k_x^2-3k_y^2)$ when magnetic
moments are $\parallel \hat{x}$. We denote its $\vec{k}$-dependence by
$f(k_x,k_y,k_z)$ and the basic case has $f=k_zk_x(k_x^2-3k_y^2)$ while
'cut-off variant' has $f=k_zk_x(k_x^2-3k_y^2)/(k_x^2+k_y^2+k_0^2)$ with
another model parameter $k_0$ (this effectively means that higher order
terms beyond $\propto k^4$ have been added in a certain specific way).

\section*{Details}

$$ H_{kp} = \frac{\hbar^2}{m_0} \begin{pmatrix}
	a k_x^2 + b k_y^2 + c k_z^2 & (a-b) k_x k_y & 0 & & &\\
	(a-b) k_x k_y & b k_x^2 + a k_y^2 + c k_z^2 & 0 & & & \\
	0 & 0 & e_z & & & \\
	& & & a k_x^2 + b k_y^2 + c k_z^2 & (a-b) k_x k_y \\
	& & & (a-b) k_x k_y & b k_x^2 + a k_y^2 + c k_z^2 & 0 \\
	& & & 0 & 0 & e_z
\end{pmatrix}$$

$$ H_{SO} = \begin{pmatrix} 
0 & 0 & 0 & 0 & it\Delta_z & t \Delta_{xy} \\
0 & 0 & -i \Delta_{xy} & -it\Delta_z & 0 & 0 \\
0 & i \Delta_{xy} & 0 & -t \Delta_{xy} & 0 & 0 \\
0 & it\Delta_z & -t \Delta_{xy} & 0 & 0 & 0 \\
-it\Delta_z & 0 & 0 & 0 & 0 & i \Delta_{xy} \\
t \Delta_{xy} & 0 & 0 & 0 & -i \Delta_{xy} & 0 \\
\end{pmatrix}$$

%$$ H_{cryst} = J k_z k_x (k_x^2-3k_y^2) (\sigma_i \otimes I_3)$$
$$ H_{cryst}= \begin{pmatrix}
J k_z k_x (k_x^2-3k_y^2) & 0 & 0 &&& \\
0 & J k_z k_x (k_x^2-3k_y^2) & 0 &&& \\
0 & 0 & 0 &&& \\
&&& -J k_z k_x (k_x^2-3k_y^2) & 0 & 0 \\
&&& 0 & -J k_z k_x (k_x^2-3k_y^2) & 0 &&& \\
&&& 0 & 0 & 0 &&&
\end{pmatrix}$$

\section*{Plan}
\begin{enumerate}
	\item Fit the Hamiltonian $H = H_{kp}$ to the ab initio bandstructure without spin-orbit coupling in the vicinity of the A-Point. In this step, $e_z$ is set to $-\inf$, so that the matrix reduces to an effective 2x2 problem, since the residual 4x4 matrix is made up of two identical 2x2 blocks. The parameters $a, b, c$ are obtained and the related effective masses are calculated.
	\item The Hamiltonian is extended by including the spin-orbit coupling (SOC): $H = H_{kp} + H_{SO}$, where $a,b,c$ are used from the previous step and $e_z = -\inf$ still holds true, which makes it an effective 4x4 problem. Due to $e_z = -\inf$, the in-plane coupling $\Delta_{xy}$ is still irrelevant (can however set to $\Delta_{xy} = x$ meV, obtained from ...). $\Delta_{z} = x$ obtained from ... . The relation $t \Delta_z$ obtained by replacing it with $\alpha k_z$, fitting $\alpha$ to the ab initio bandstructure \textit{with} spin-orbit coupling and then calculating $t$ with the known $\alpha$ and $\Delta_z$. It holds that $t \propto k_z$.
	\item $H_{cryst}$ is included and $J$ obtained by fitting to another bandstructure. This term shall explain the deviation of the maximum from the A-point (?).
	\item $e_z$ is set to a finite value, which makes it an effective 6x6 problem and $\Delta_{xy}$ obtained by fitting the new emerging band to the bandstructure.
\end{enumerate}

\begin{thebibliography}{99}
  \bibitem{PFjr} arXiv {\tt 2204.04206}
  \bibitem{LandT} arXiv {\tt 2105.05820}
\end{thebibliography}

\end{document}

